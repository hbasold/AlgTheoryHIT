\documentclass{easychair}

\usepackage[T1]{fontenc}
\usepackage[utf8]{inputenc}

\usepackage{amsmath,amssymb,mathtools}
\usepackage{xparse}
\usepackage{url}
\usepackage{agda}

\DeclareUnicodeCharacter{1D4E4}{\ensuremath{\mathcal{U}}}
\DeclareUnicodeCharacter{1D4E5}{\ensuremath{\mathcal{V}}}
\DeclareUnicodeCharacter{1D4E6}{\ensuremath{\mathcal{W}}}
\DeclareUnicodeCharacter{2294}{\ensuremath{\sqcup}}
\DeclareUnicodeCharacter{207A}{\ensuremath{^{+}}}
\DeclareUnicodeCharacter{0307}{}
\DeclareUnicodeCharacter{2200}{\ensuremath{\forall}}
\DeclareUnicodeCharacter{2192}{\ensuremath{\to}}
\DeclareUnicodeCharacter{2080}{\ensuremath{_{0}}}
\DeclareUnicodeCharacter{03A3}{\ensuremath{\Sigma}}
\DeclareUnicodeCharacter{03B1}{\ensuremath{\alpha}}
\DeclareUnicodeCharacter{03BB}{\ensuremath{\lambda}}

\DeclareDocumentCommand{\sig}{}{\AgdaBound{Σ}}
\DeclareDocumentCommand{\ofT}{}{\AgdaSpace{}\AgdaSymbol{:}\AgdaSpace{}}
\DeclareDocumentCommand{\toA}{}{\AgdaSpace{}\AgdaSymbol{$\to$}\AgdaSpace{}}
\DeclareDocumentCommand{\U}{m}{\AgdaBound{$\mathcal{#1}$}}
\DeclareDocumentCommand{\UZ}{m}{\AgdaPrimitive{$\mathcal{U}_{0}$}}
\DeclareDocumentCommand{\UO}{m}{\AgdaPrimitive{$\mathcal{U}_{1}$}}
\DeclareDocumentCommand{\Sig}{m m}{%
  \AgdaRecord{Signature}%
}
\DeclareDocumentCommand{\Term}{m m}{%
  \AgdaDatatype{Term}\AgdaSpace{}%
  #1\AgdaSpace{}%
  #2%
}
\DeclareDocumentCommand{\FreeAlg}{m}{%
  \AgdaDatatype{FreeAlgebra}\AgdaSpace{}#1%
}

\title{Free Algebraic Theories as Higher Inductive Types}

\author{
  Henning Basold\inst{1}\thanks{This work is supported by the European Research
    Council (ERC) under the EU’s Horizon 2020 programme~(CoVeCe, grant
    agreement No. 678157)}
\and
  Niels van der Weide\inst{2}
\and
  Niccolò Veltri\inst{3}
}

% Institutes for affiliations are also joined by \and,
\institute{
  CNRS, ENS Lyon \\
  \email{henning.basold@ens-lyon.fr}
\and
   Radboud University Nijmegen\\
   \email{nweide@cs.ru.nl}
\and
   IT University of Copenhagen\\
   \email{nive@itu.dk}
}

\authorrunning{Basold, Veltri and van der Weide}
% \titlerunning{The {\easychair} Class File}


\begin{document}

\maketitle

In recent years, there has been an increasing interest in higher inductive
types.
There are several reasons behind this, like synthetic homotopy
theory~\cite{hottbook},
implementation of rewrite rules~\cite{Altenkirch16:TTinTTusingQIT},
quotients~\cite{Basold16:HIT-Prog},
and other colimits.
Here we are interested in applications of higher inductive types
in universal algebra and category theory, and the possibility of extending inductive and
coinductive types beyond strictly positive types.
% Let us briefly explain this here.
One of the most basic constructions in universal algebra is that of a free
algebra.
This construction can be carried out by using higher inductive types, as we will
now briefly show.

We present some basic notions as Agda code.
In what follows, we will work in Martin-Löf type theory with two basic
universes \UZ{}\ofT\UO{}.
We will also require function extensionality, which can, however, be dropped
if we restrict ourselves to finitary signature.
The following record type defines a \emph{signature} (also polynomial or
container) in Agda.
\begin{code}%
\>[0]\AgdaKeyword{record}\AgdaSpace{}%
\AgdaRecord{Signature} \ofT \UO%
\AgdaOperator{\AgdaFunction{̇}}\AgdaSpace{}%
\AgdaKeyword{where}\<%
\\
% \>[0][@{}l@{\AgdaIndent{0}}]%
% \>[2]\AgdaKeyword{field}\<%
% \\
\>[2][@{}l@{\AgdaIndent{0}}]%
\>[4]\AgdaField{sym}%
\>[9]\AgdaSymbol{:}\AgdaSpace{}%
\UZ\AgdaSpace{}%
\AgdaOperator{\AgdaFunction{̇}}\<%
\\
%
\>[4]\AgdaField{ar}%
\>[9]\AgdaSymbol{:}\AgdaSpace{}%
\AgdaField{sym}\AgdaSpace{}%
\AgdaSymbol{→}\AgdaSpace{}%
\UZ\AgdaSpace{}%
\AgdaOperator{\AgdaFunction{̇}}\<%
% \\
% \>[0]\AgdaKeyword{open}\AgdaSpace{}%
% \AgdaModule{Signature}\AgdaSpace{}%
% \AgdaKeyword{public}\<%
\end{code}
From a signature \sig \ofT \Sig{U}{V}, we can construct the type
\Term{\sig}{X} of terms (W-type) over \sig{} with leaves labelled in X.
This type comes with the obvious iteration principle that we denote by
\AgdaFunction{Term-iter}.
An \emph{algebraic theory} is given by a signature and a set of equations
between terms over that signature.
These equations may have free variables but may not use other properties than
equality on the variables.
Thus, we represent equations as parametric binary
relations~\cite{Hermida14:LogicalRelationsParametricity}, as in the following
record type.
\begin{code}%
\>[0]\AgdaKeyword{record}\AgdaSpace{}%
\AgdaRecord{AlgTheory}\AgdaSpace{}%
\AgdaSymbol{:}\AgdaSpace{}%
\UO%
\AgdaOperator{\AgdaFunction{̇}}\AgdaSpace{}%
\AgdaKeyword{where}\<%
\\
% \>[0][@{}l@{\AgdaIndent{0}}]%
% \>[2]\AgdaKeyword{field}\<%
% \\
\>[2][@{}l@{\AgdaIndent{0}}]%
\>[4]\AgdaField{sig}%
\>[9]\AgdaSymbol{:}\AgdaSpace{}%
\AgdaRecord{Signature}\<%
\\
%
\>[4]\AgdaField{eqs}%
\>[9]\AgdaSymbol{:}\AgdaSpace{}%
\AgdaSymbol{∀}\AgdaSpace{}%
\AgdaSymbol{\{}\AgdaBound{X}\AgdaSpace{}%
\AgdaSymbol{:}\AgdaSpace{}%
\UZ\AgdaSpace{}%
\AgdaOperator{\AgdaFunction{̇}}\AgdaSymbol{\}}\AgdaSpace{}%
\AgdaSymbol{→}\AgdaSpace{}%
\AgdaFunction{Rel}\AgdaSpace{}%
\AgdaSymbol{(}\AgdaDatatype{Term}\AgdaSpace{}%
\AgdaField{sig}\AgdaSpace{}%
\AgdaBound{X}\AgdaSymbol{)}\<%
\end{code}
Note that the requirement that the variable type \AgdaBound{X} is in a fixed
universe \UZ, will also fix the universe in which the algebras for
a theory live.
This is, however, not a severe constraint, as the induction principle still
allows for large elimination.

Given an algebraic theory
\AgdaBound{T} \ofT \AgdaRecord{AlgTheory}
over a signature \AgdaBound{Σ}, we can define what an algebra and an induction
scheme for an algebra of \AgdaBound{T} is.
Algebras are given in two steps: First, we define pre-algebras that do not have
to fulfil the equations of the theory, but only are an algebra for the functor
induced by the signature Σ.
The fact that the carrier of a pre-algebra is a set (in the sense of homotopy
type theory) is technical necessity, which we would like to lift in the future.
Note that this pre-algebra immediately extends to all terms by freeness.
An actual algebra for T is then a pre-algebra that fulfils also the equations
required by the theory.
This idea is formalised in the following two record types.
\newpage
\begin{code}%
\>[0][@{}l@{\AgdaIndent{1}}]%
\>[2]\AgdaKeyword{record}\AgdaSpace{}%
\AgdaRecord{PreAlgebra}\AgdaSpace{}%
\AgdaSymbol{:}\AgdaSpace{}%
\UO
\AgdaOperator{\AgdaFunction{̇}}\AgdaSpace{}%
\AgdaKeyword{where}\<%
\\
\>[2][@{}l@{\AgdaIndent{0}}]%
% \>[4]\AgdaKeyword{field}\<%
% \\
\>[4][@{}l@{\AgdaIndent{0}}]%
\>[6]\AgdaField{carrier}%
\>[19]\AgdaSymbol{:}\AgdaSpace{}%
\UZ\AgdaSpace{}%
\AgdaOperator{\AgdaFunction{̇}}\<%
\\
\>[4][@{}l@{\AgdaIndent{0}}]%
\>[6]\AgdaField{carrier-set}%
\>[19]\AgdaSymbol{:}\AgdaSpace{}%
\AgdaFunction{is-set}\AgdaSpace{}%
\AgdaFunction{carrier}\<%
\\
%
\>[6]\AgdaField{algebra}%
\>[15]\AgdaSymbol{:}\AgdaSpace{}%
\AgdaSymbol{(}\AgdaBound{s}\AgdaSpace{}%
\AgdaSymbol{:}\AgdaSpace{}%
\AgdaField{sym}\AgdaSpace{}\AgdaBound{Σ}\AgdaSymbol{)}\AgdaSpace{}%
\AgdaSymbol{(}\AgdaBound{α}\AgdaSpace{}%
\AgdaSymbol{:}\AgdaSpace{}%
\AgdaField{ar}\AgdaSpace{}%
\AgdaBound{Σ}\AgdaSpace{}%
\AgdaBound{s}\AgdaSpace{}%
\AgdaSymbol{→}\AgdaSpace{}%
\AgdaField{carrier}\AgdaSymbol{)}\AgdaSpace{}%
\AgdaSymbol{→}\AgdaSpace{}%
\AgdaField{carrier}\<%
\\
%
\\[\AgdaEmptyExtraSkip]%
%
\>[4]\AgdaFunction{algebra*}\AgdaSpace{}%
\AgdaSymbol{:}\AgdaSpace{}%
\AgdaDatatype{Term}\AgdaSpace{}%
\AgdaBound{Σ}\AgdaSpace{}%
\AgdaField{carrier}\AgdaSpace{}%
\AgdaSymbol{→}\AgdaSpace{}%
\AgdaField{carrier}\<%
\\
%
\>[4]\AgdaFunction{algebra*}\AgdaSpace{}%
\AgdaSymbol{=}\AgdaSpace{}%
\AgdaFunction{Term-iter}\AgdaSpace{}%
\AgdaSymbol{(λ}\AgdaSpace{}%
\AgdaBound{x}\AgdaSpace{}%
\AgdaSymbol{→}\AgdaSpace{}%
\AgdaBound{x}\AgdaSymbol{)}\AgdaSpace{}%
\AgdaField{algebra}\<%
\\%
%
\\[\AgdaEmptyExtraSkip]%
%
\>[0][@{}l@{\AgdaIndent{1}}]%
\>[2]\AgdaKeyword{record}\AgdaSpace{}%
\AgdaRecord{Algebra}\AgdaSpace{}%
\AgdaSymbol{:}\AgdaSpace{}%
\UO
\AgdaOperator{\AgdaFunction{̇}}\AgdaSpace{}%
\AgdaKeyword{where}\<%
\\
\>[2][@{}l@{\AgdaIndent{0}}]%
% \>[4]\AgdaKeyword{field}\<%
% \\
\>[4][@{}l@{\AgdaIndent{0}}]%
\>[6]\AgdaField{pre-algebra}%
\>[19]\AgdaSymbol{:}\AgdaSpace{}%
\AgdaRecord{PreAlgebra}\<%
\\
%
% \>[4]\AgdaKeyword{open}\AgdaSpace{}%
% \AgdaModule{PreAlgebra}\AgdaSpace{}%
% \AgdaField{pre-algebra}\AgdaSpace{}%
% \AgdaKeyword{public}\<%
% \\
%
% \>[4]\AgdaKeyword{field}\<%
% \\
%
\>[6]\AgdaField{resp-eq}%
\>[19]\AgdaSymbol{:}%
\>[417I]\AgdaSymbol{∀}\AgdaSpace{}%
\AgdaSymbol{\{}\AgdaBound{t}\AgdaSpace{}%
\AgdaBound{u}\AgdaSpace{}%
\AgdaSymbol{:}\AgdaSpace{}%
\AgdaDatatype{Term}\AgdaSpace{}%
\sig\AgdaSpace{}%
\AgdaFunction{carrier}\AgdaSymbol{\}}\AgdaSpace{}%
\AgdaSymbol{→}\<%
\\
\>[.][@{}l@{}]\<[417I]%
\>[21]\AgdaField{eqs}\AgdaSpace{}%
\AgdaBound{t}\AgdaSpace{}%
\AgdaBound{u}\AgdaSpace{}%
\AgdaSymbol{→}\AgdaSpace{}%
\AgdaFunction{algebra*}\AgdaSpace{}%
\AgdaBound{t}\AgdaSpace{}%
\AgdaOperator{\AgdaDatatype{==}}\AgdaSpace{}%
\AgdaFunction{algebra*}\AgdaSpace{}%
\AgdaBound{u}\<%
\end{code}

One can then also define an induction scheme for algebras of an algebraic
theory, cf.~\cite{HermidaJacobs97:IndCoindFib}.
With the basic definitions in place, we construct the initial algebra of
an algebraic theory as higher inductive type with the following
(type) constructors, where the function
\AgdaFunction{node*} of type
\Term{\sig}{\AgdaSymbol{(}\FreeAlg{\AgdaBound{T}\AgdaSymbol{)}}}
\toA \FreeAlg{\AgdaBound{T}}
is the extension of \AgdaFunction{node} to terms,
cf. \AgdaFunction{algebra*} above.
\begin{code}%
% \>[0]\AgdaKeyword{module}\AgdaSpace{}%
% \AgdaModule{\AgdaUnderscore{}}\AgdaSpace{}%
% \AgdaKeyword{where}\<%
% \\
% \>[0][@{}l@{\AgdaIndent{0}}]%
% \>[2]\AgdaKeyword{postulate}\AgdaSpace{}%
% \AgdaComment{-- HIT type}\<%
% \\
% \>[2][@{}l@{\AgdaIndent{0}}]%
\>[6]\AgdaPostulate{FreeAlgebra}\AgdaSpace{}%
\>[13]\AgdaSymbol{:}\AgdaSpace{}%
\AgdaSymbol{(}\AgdaBound{T}\AgdaSpace{}%
\AgdaSymbol{:}\AgdaSpace{}%
\AgdaRecord{AlgTheory}\AgdaSymbol{)}\AgdaSpace{}%
\AgdaSymbol{→}\AgdaSpace{}%
\UZ\AgdaSpace{}%
\AgdaOperator{\AgdaFunction{̇}}\<%
\\
%
% \\[\AgdaEmptyExtraSkip]%
% %
% \>[2]\AgdaKeyword{module}\AgdaSpace{}%
% \AgdaModule{\AgdaUnderscore{}}\AgdaSpace{}%
% \AgdaSymbol{\{}\AgdaBound{𝓤}\AgdaSymbol{\}}\AgdaSpace{}%
% \AgdaSymbol{\{}\AgdaBound{T}\AgdaSpace{}%
% \AgdaSymbol{:}\AgdaSpace{}%
% \AgdaRecord{AlgTheory}\AgdaSpace{}%
% \AgdaPrimitive{𝓤₀}\AgdaSpace{}%
% \AgdaPrimitive{𝓤₀}\AgdaSpace{}%
% \AgdaBound{𝓤}\AgdaSymbol{\}}\AgdaSpace{}%
% \AgdaKeyword{where}\<%
% \\
% \>[2][@{}l@{\AgdaIndent{0}}]%
% \>[4]\AgdaKeyword{open}\AgdaSpace{}%
% \AgdaModule{AlgTheory}\AgdaSpace{}%
% \AgdaBound{T}\AgdaSpace{}%
% \AgdaKeyword{renaming}\AgdaSpace{}%
% \AgdaSymbol{(}sig \AgdaSymbol{to} Σ\AgdaSymbol{)}\<%
% \\
% %
% \\[\AgdaEmptyExtraSkip]%
% %
% \>[4]\AgdaKeyword{postulate}\AgdaSpace{}%
% \AgdaComment{-- HIT 0-constructor}\<%
% \\
% \>[4][@{}l@{\AgdaIndent{0}}]%
\>[6]\AgdaPostulate{node}%
\>[13]\AgdaSymbol{:}\AgdaSpace{}%
\AgdaSymbol{(}\AgdaBound{s}\AgdaSpace{}%
\AgdaSymbol{:}\AgdaSpace{}%
\AgdaField{sym}\AgdaSpace{}%
\sig\AgdaSymbol{)}\AgdaSpace{}%
\AgdaSymbol{(}\AgdaBound{α}\AgdaSpace{}%
\AgdaSymbol{:}\AgdaSpace{}%
\AgdaField{ar}\AgdaSpace{}%
\sig\AgdaSpace{}%
\AgdaBound{s}\AgdaSpace{}%
\AgdaSymbol{→}\AgdaSpace{}%
\AgdaPostulate{FreeAlgebra}\AgdaSpace{}%
\AgdaBound{T}\AgdaSymbol{)}\AgdaSpace{}%
\AgdaSymbol{→}\AgdaSpace{}%
\AgdaPostulate{FreeAlgebra}\AgdaSpace{}%
\AgdaBound{T}\<%
\\
\>[6]\AgdaPostulate{eq}%
\>[13]\AgdaSymbol{:}\AgdaSpace{}%
\AgdaSymbol{∀}\AgdaSpace{}%
\AgdaSymbol{\{}\AgdaBound{t}\AgdaSpace{}%
\AgdaBound{u}\AgdaSymbol{\}}\AgdaSpace{}%
\AgdaSymbol{→}\AgdaSpace{}%
\AgdaField{eqs}\AgdaSpace{}%
\AgdaBound{t}\AgdaSpace{}%
\AgdaBound{u}\AgdaSpace{}%
\AgdaSymbol{→}\AgdaSpace{}%
\AgdaFunction{node*}\AgdaSpace{}%
\AgdaBound{t}\AgdaSpace{}%
\AgdaOperator{\AgdaDatatype{==}}\AgdaSpace{}%
\AgdaFunction{node*}\AgdaSpace{}%
\AgdaBound{u}\<%
\\
%
\>[6]\AgdaPostulate{quot}%
\>[13]\AgdaSymbol{:}\AgdaSpace{}%
\AgdaFunction{is-set}\AgdaSpace{}%
\AgdaSymbol{(}\AgdaPostulate{FreeAlgebra}\AgdaSpace{}%
\AgdaBound{T}\AgdaSymbol{)}\<%
\end{code}

This HIT comes with an iteration principle for \AgdaBound{T}-algebras and an
induction principle.
These can be used to show that \FreeAlg{\AgdaBound{T}} is indeed the initial
\AgdaBound{T}-algebra.
Details and examples can be found here:
\url{https://perso.ens-lyon.fr/henning.basold/code/AlgTheoryHIT}.

Why would we be interested in such a construction of free algebras?
First of all, we wish to formalise parts of universal algebras, like
quotient algebras and the isomorphism theorems.
Higher inductive types seem to provide the appropriate mechanism for such
an endeavour.
Moreover, once we can construct free algebras, we could also consider inductive
and coinductive types over algebraic theories as an extension of strictly
positive types.
In particular, semantics of finitely branching transition systems could be
obtained in the final coalgebra for the finite powerset functor, which is
the free join-semilattice and therefore representable in our framework.
Finally, we aim to extend the construction to HITs that are not just sets, but
groupoids.
This would enable us, for example, to construct free symmetric monoidal
categories.

In the talk, we will present the algebras and the induction scheme in more
detail, show applications of the above construction, and discuss future
directions.

% Future work:
% \begin{itemize}
% \item Adjunction between sets and algebras that factors through quotient for
%   finitary signatures (lifting quotient to algebras?)
% \item Wider class of inductive/coinductive types; allow free algebra is
%   input/output type
% \item Move from sets to groupoids (free symmetric monoidal category example)
% \end{itemize}

%\bibliographystyle{plain}
%\bibliographystyle{alpha}
%\bibliographystyle{unsrt}
\bibliographystyle{abbrv}
\bibliography{AlgTheoriesHIT}

\end{document}
