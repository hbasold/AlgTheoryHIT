\documentclass{easychair}

\usepackage[T1]{fontenc}
\usepackage[utf8]{inputenc}

\usepackage{amsmath,amssymb,mathtools}
\usepackage{xparse}
\usepackage{url}
\usepackage{agda}

\DeclareUnicodeCharacter{1D4E4}{\ensuremath{\mathcal{U}}}
\DeclareUnicodeCharacter{1D4E5}{\ensuremath{\mathcal{V}}}
\DeclareUnicodeCharacter{1D4E6}{\ensuremath{\mathcal{W}}}
\DeclareUnicodeCharacter{2294}{\ensuremath{\sqcup}}
\DeclareUnicodeCharacter{207A}{\ensuremath{^{+}}}
\DeclareUnicodeCharacter{0307}{}
\DeclareUnicodeCharacter{2200}{\ensuremath{\forall}}
\DeclareUnicodeCharacter{2192}{\ensuremath{\to}}
\DeclareUnicodeCharacter{2080}{\ensuremath{_{0}}}
\DeclareUnicodeCharacter{03A3}{\ensuremath{\Sigma}}
\DeclareUnicodeCharacter{03B1}{\ensuremath{\alpha}}
\DeclareUnicodeCharacter{03BB}{\ensuremath{\lambda}}

\DeclareDocumentCommand{\sig}{}{\AgdaBound{Σ}}
\DeclareDocumentCommand{\ofT}{}{\AgdaSpace{}\AgdaSymbol{:}\AgdaSpace{}}
\DeclareDocumentCommand{\toA}{}{\AgdaSpace{}\AgdaSymbol{$\to$}\AgdaSpace{}}
\DeclareDocumentCommand{\U}{m}{\AgdaBound{$\mathcal{#1}$}}
\DeclareDocumentCommand{\Sig}{m m}{%
  \AgdaRecord{Signature}\AgdaSpace{}%
  \U{#1}\AgdaSpace{}\U{#2}%
}
\DeclareDocumentCommand{\Term}{m m}{%
  \AgdaDatatype{Term}\AgdaSpace{}%
  #1\AgdaSpace{}%
  #2%
}
\DeclareDocumentCommand{\FreeAlg}{m}{%
  \AgdaDatatype{FreeAlgebra}\AgdaSpace{}#1%
}

\title{Free Algebraic Theories as Higher Inductive Types}

\author{
  Henning Basold\inst{1}% \thanks{Grant}
\and
  Niccolò Veltri\inst{2}
\and
  Niels van der Weide\inst{3}
}

% Institutes for affiliations are also joined by \and,
\institute{
  CNRS, ENS Lyon \\
  \email{henning.basold@ens-lyon.fr}
\and
   IT University of Copenhagen\\
   \email{nive@itu.dk}
\and
   Radboud University Nijmegen\\
   \email{N.vanderWeide@science.ru.nl}
}

\authorrunning{Basold, Veltri and van der Weide}
% \titlerunning{The {\easychair} Class File}


\begin{document}

\maketitle

In recent years, there has been an increasing interest in higher inductive
types.
There are several reasons for this like synthetic homotopy
theory~\cite{hottbook},
implementation of rewrite rules~\cite{Altenkirch16:TTinTTusingQIT},
quotients~\cite{Basold16:HIT-Prog},
and other colimits.
The reason we are interested in higher inductive types here is their application
in universal algebra and category theory, and the extension of inductive and
coinductive types beyond strictly positive types.
% Let us briefly explain this here.
One of the most basic constructions in universal algebra is that of a free
algebra.
This construction can be carried out by using higher inductive types, as we will
now briefly show.

We present some basic notions as Agda code.
Here, and in what follows, \U{U}, \U{V} and \U{W}
denote universes, \AgdaPrimitive{⊔} is the least upper bound of universes,
and \U{U}\AgdaPrimitive{⁺} denotes the next universe above \U{U}.
The following record type defines a \emph{signature} (also polynomial or
container) in Agda.
\begin{code}%
\>[0]\AgdaKeyword{record}\AgdaSpace{}%
\AgdaRecord{Signature}\AgdaSpace{}%
\AgdaSymbol{(}\AgdaBound{𝓤}\AgdaSpace{}%
\AgdaBound{𝓥}\AgdaSymbol{)}\AgdaSpace{}%
\AgdaSymbol{:}\AgdaSpace{}%
\AgdaSymbol{(}\AgdaBound{𝓤}\AgdaSpace{}%
\AgdaOperator{\AgdaPrimitive{⊔}}\AgdaSpace{}%
\AgdaBound{𝓥}\AgdaSymbol{)}\AgdaSpace{}%
\AgdaOperator{\AgdaPrimitive{⁺}}\AgdaSpace{}%
\AgdaOperator{\AgdaFunction{̇}}\AgdaSpace{}%
\AgdaKeyword{where}\<%
\\
% \>[0][@{}l@{\AgdaIndent{0}}]%
% \>[2]\AgdaKeyword{field}\<%
% \\
\>[2][@{}l@{\AgdaIndent{0}}]%
\>[4]\AgdaField{sym}%
\>[9]\AgdaSymbol{:}\AgdaSpace{}%
\AgdaBound{𝓤}\AgdaSpace{}%
\AgdaOperator{\AgdaFunction{̇}}\<%
\\
%
\>[4]\AgdaField{ar}%
\>[9]\AgdaSymbol{:}\AgdaSpace{}%
\AgdaField{sym}\AgdaSpace{}%
\AgdaSymbol{→}\AgdaSpace{}%
\AgdaBound{𝓥}\AgdaSpace{}%
\AgdaOperator{\AgdaFunction{̇}}\<%
% \\
% \>[0]\AgdaKeyword{open}\AgdaSpace{}%
% \AgdaModule{Signature}\AgdaSpace{}%
% \AgdaKeyword{public}\<%
\end{code}
From a signature \sig \ofT \Sig{U}{V}, we can construct the type
\Term{\sig}{X} of terms (W-type) over \sig{} with leaves labelled in X.
This type comes with the obvious iteration principle that we denote by
\AgdaFunction{Term-iter}.
An \emph{algebraic theory} is given by a signature and a set of equations
between terms over that signature.
These equations may have free variables by may not use other properties than
equality on the variables.
Thus, we represent equations as parametric
relations~\cite{Hermida14:LogicalRelationsParametricity}, as in the following
record type.
\begin{code}%
\>[0]\AgdaKeyword{record}\AgdaSpace{}%
\AgdaRecord{AlgTheory}\AgdaSpace{}%
\AgdaSymbol{(}\AgdaBound{𝓤}\AgdaSpace{}%
\AgdaBound{𝓥}\AgdaSpace{}%
\AgdaBound{𝓦}\AgdaSymbol{)}\AgdaSpace{}%
\AgdaSymbol{:}\AgdaSpace{}%
\AgdaSymbol{(}\AgdaBound{𝓤}\AgdaSpace{}%
\AgdaOperator{\AgdaPrimitive{⊔}}\AgdaSpace{}%
\AgdaBound{𝓥}\AgdaSpace{}%
\AgdaOperator{\AgdaPrimitive{⊔}}\AgdaSpace{}%
\AgdaBound{𝓦}\AgdaSymbol{)}\AgdaSpace{}%
\AgdaOperator{\AgdaPrimitive{⁺}}\AgdaSpace{}%
\AgdaOperator{\AgdaFunction{̇}}\AgdaSpace{}%
\AgdaKeyword{where}\<%
\\
% \>[0][@{}l@{\AgdaIndent{0}}]%
% \>[2]\AgdaKeyword{field}\<%
% \\
\>[2][@{}l@{\AgdaIndent{0}}]%
\>[4]\AgdaField{sig}%
\>[9]\AgdaSymbol{:}\AgdaSpace{}%
\AgdaRecord{Signature}\AgdaSpace{}%
\AgdaBound{𝓤}\AgdaSpace{}%
\AgdaBound{𝓥}\<%
\\
%
\>[4]\AgdaField{eqs}%
\>[9]\AgdaSymbol{:}\AgdaSpace{}%
\AgdaSymbol{∀}\AgdaSpace{}%
\AgdaSymbol{\{}\AgdaBound{X}\AgdaSpace{}%
\AgdaSymbol{:}\AgdaSpace{}%
\AgdaBound{𝓦}\AgdaSpace{}%
\AgdaOperator{\AgdaFunction{̇}}\AgdaSymbol{\}}\AgdaSpace{}%
\AgdaSymbol{→}\AgdaSpace{}%
\AgdaFunction{Rel}\AgdaSpace{}%
\AgdaSymbol{(}\AgdaDatatype{Term}\AgdaSpace{}%
\AgdaField{sig}\AgdaSpace{}%
\AgdaBound{X}\AgdaSymbol{)}\AgdaSpace{}%
\AgdaPrimitive{𝓤₀}\<%
\end{code}
Note that the requirement that the variable type \AgdaBound{X} is in a fixed
universe \AgdaBound{𝓦}, will also fix the universe in which the algebras for
a theory live.
This is, however, not a severe constraint, as the induction principle still
allows for large elimination.

We fix universes \U{U}, \U{V} and \U{W}.
Given an algebraic theory
\AgdaBound{T}\AgdaSpace{}%
\AgdaSymbol{:}\AgdaSpace{}%
\AgdaRecord{AlgTheory}\AgdaSpace{}%
\AgdaBound{𝓤}\AgdaSpace{}%
\AgdaBound{𝓥}\AgdaSpace{}%
\AgdaBound{𝓦}
over a signature \AgdaBound{Σ}, we can define what an algebra and an induction
scheme for an algebra of \AgdaBound{T} is.
Algebras are given in two steps: First, we define pre-algebras that do not have
to fulfil the equations of the theory, but only are an algebra for the functor
induced by the signature Σ.
Note that this pre-algebra immediately extends to all terms by freeness.
\begin{code}%
\>[0][@{}l@{\AgdaIndent{1}}]%
\>[2]\AgdaKeyword{record}\AgdaSpace{}%
\AgdaRecord{PreAlgebra}\AgdaSpace{}%
\AgdaSymbol{:}\AgdaSpace{}%
\AgdaSymbol{(}\AgdaBound{𝓤}\AgdaSpace{}%
\AgdaOperator{\AgdaPrimitive{⊔}}\AgdaSpace{}%
\AgdaBound{𝓥}\AgdaSpace{}%
\AgdaOperator{\AgdaPrimitive{⊔}}\AgdaSpace{}%
\AgdaBound{𝓦}\AgdaSymbol{)}\AgdaSpace{}%
\AgdaOperator{\AgdaPrimitive{⁺}}\AgdaSpace{}%
\AgdaOperator{\AgdaFunction{̇}}\AgdaSpace{}%
\AgdaKeyword{where}\<%
\\
\>[2][@{}l@{\AgdaIndent{0}}]%
% \>[4]\AgdaKeyword{field}\<%
% \\
\>[4][@{}l@{\AgdaIndent{0}}]%
\>[6]\AgdaField{carrier}%
\>[19]\AgdaSymbol{:}\AgdaSpace{}%
\AgdaBound{𝓦}\AgdaSpace{}%
\AgdaOperator{\AgdaFunction{̇}}\<%
\\
%
\>[6]\AgdaField{algebra}%
\>[15]\AgdaSymbol{:}\AgdaSpace{}%
\AgdaSymbol{(}\AgdaBound{s}\AgdaSpace{}%
\AgdaSymbol{:}\AgdaSpace{}%
\AgdaField{sym}\AgdaSpace{}\AgdaBound{Σ}\AgdaSymbol{)}\AgdaSpace{}%
\AgdaSymbol{(}\AgdaBound{α}\AgdaSpace{}%
\AgdaSymbol{:}\AgdaSpace{}%
\AgdaField{ar}\AgdaSpace{}%
\AgdaBound{Σ}\AgdaSpace{}%
\AgdaBound{s}\AgdaSpace{}%
\AgdaSymbol{→}\AgdaSpace{}%
\AgdaField{carrier}\AgdaSymbol{)}\AgdaSpace{}%
\AgdaSymbol{→}\AgdaSpace{}%
\AgdaField{carrier}\<%
\\
%
\\[\AgdaEmptyExtraSkip]%
%
\>[4]\AgdaFunction{algebra*}\AgdaSpace{}%
\AgdaSymbol{:}\AgdaSpace{}%
\AgdaDatatype{Term}\AgdaSpace{}%
\AgdaBound{Σ}\AgdaSpace{}%
\AgdaField{carrier}\AgdaSpace{}%
\AgdaSymbol{→}\AgdaSpace{}%
\AgdaField{carrier}\<%
\\
%
\>[4]\AgdaFunction{algebra*}\AgdaSpace{}%
\AgdaSymbol{=}\AgdaSpace{}%
\AgdaFunction{Term-iter}\AgdaSpace{}%
\AgdaSymbol{(λ}\AgdaSpace{}%
\AgdaBound{x}\AgdaSpace{}%
\AgdaSymbol{→}\AgdaSpace{}%
\AgdaBound{x}\AgdaSymbol{)}\AgdaSpace{}%
\AgdaField{algebra}\<%
\end{code}
An actual algebra for T is then a pre-algebra whose carrier is a set (in the
sense of homotopy type theory) and that fulfils also the equations required by
the theory.
\begin{code}%
\>[0][@{}l@{\AgdaIndent{1}}]%
\>[2]\AgdaKeyword{record}\AgdaSpace{}%
\AgdaRecord{Algebra}\AgdaSpace{}%
\AgdaSymbol{:}\AgdaSpace{}%
\AgdaSymbol{(}\AgdaBound{𝓤}\AgdaSpace{}%
\AgdaOperator{\AgdaPrimitive{⊔}}\AgdaSpace{}%
\AgdaBound{𝓥}\AgdaSpace{}%
\AgdaOperator{\AgdaPrimitive{⊔}}\AgdaSpace{}%
\AgdaBound{𝓦}\AgdaSymbol{)}\AgdaSpace{}%
\AgdaOperator{\AgdaPrimitive{⁺}}\AgdaSpace{}%
\AgdaOperator{\AgdaFunction{̇}}\AgdaSpace{}%
\AgdaKeyword{where}\<%
\\
\>[2][@{}l@{\AgdaIndent{0}}]%
% \>[4]\AgdaKeyword{field}\<%
% \\
\>[4][@{}l@{\AgdaIndent{0}}]%
\>[6]\AgdaField{pre-algebra}%
\>[19]\AgdaSymbol{:}\AgdaSpace{}%
\AgdaRecord{PreAlgebra}\<%
\\
%
\>[4]\AgdaKeyword{open}\AgdaSpace{}%
\AgdaModule{PreAlgebra}\AgdaSpace{}%
\AgdaField{pre-algebra}\AgdaSpace{}%
\AgdaKeyword{public}\<%
\\
%
% \>[4]\AgdaKeyword{field}\<%
% \\
\>[4][@{}l@{\AgdaIndent{0}}]%
\>[6]\AgdaField{carrier-set}%
\>[19]\AgdaSymbol{:}\AgdaSpace{}%
\AgdaFunction{is-set}\AgdaSpace{}%
\AgdaFunction{carrier}\<%
\\
%
\>[6]\AgdaField{resp-eq}%
\>[19]\AgdaSymbol{:}%
\>[417I]\AgdaSymbol{∀}\AgdaSpace{}%
\AgdaSymbol{\{}\AgdaBound{t}\AgdaSpace{}%
\AgdaBound{u}\AgdaSpace{}%
\AgdaSymbol{:}\AgdaSpace{}%
\AgdaDatatype{Term}\AgdaSpace{}%
\sig\AgdaSpace{}%
\AgdaFunction{carrier}\AgdaSymbol{\}}\AgdaSpace{}%
\AgdaSymbol{→}\<%
\\
\>[.][@{}l@{}]\<[417I]%
\>[21]\AgdaField{eqs}\AgdaSpace{}%
\AgdaBound{t}\AgdaSpace{}%
\AgdaBound{u}\AgdaSpace{}%
\AgdaSymbol{→}\AgdaSpace{}%
\AgdaFunction{algebra*}\AgdaSpace{}%
\AgdaBound{t}\AgdaSpace{}%
\AgdaOperator{\AgdaDatatype{==}}\AgdaSpace{}%
\AgdaFunction{algebra*}\AgdaSpace{}%
\AgdaBound{u}\<%
\end{code}

One can then also define an induction scheme for algebras of an algebraic
theory, cf.~\cite{HermidaJacobs97:IndCoindFib}.
With all the basic definitions in place, construct the initial algebra of
an algebraic theory as higher inductive type as follows, where
\AgdaFunction{node*} \ofT
\Term{\sig}{\AgdaSymbol{(}\FreeAlg{\AgdaBound{T}\AgdaSymbol{)}}}
\toA \FreeAlg{\AgdaBound{T}}
is the extension of \AgdaFunction{node} to terms,
cf. \AgdaFunction{algebra*} above.
\begin{code}%
% \>[0]\AgdaKeyword{module}\AgdaSpace{}%
% \AgdaModule{\AgdaUnderscore{}}\AgdaSpace{}%
% \AgdaKeyword{where}\<%
% \\
% \>[0][@{}l@{\AgdaIndent{0}}]%
% \>[2]\AgdaKeyword{postulate}\AgdaSpace{}%
% \AgdaComment{-- HIT type}\<%
% \\
% \>[2][@{}l@{\AgdaIndent{0}}]%
\>[6]\AgdaPostulate{FreeAlgebra}\AgdaSpace{}%
\>[13]\AgdaSymbol{:}\AgdaSpace{}%
\AgdaSymbol{∀}\AgdaSpace{}%
\AgdaSymbol{\{}\AgdaBound{𝓤}\AgdaSymbol{\}}\AgdaSpace{}%
\AgdaSymbol{(}\AgdaBound{T}\AgdaSpace{}%
\AgdaSymbol{:}\AgdaSpace{}%
\AgdaRecord{AlgTheory}\AgdaSpace{}%
\AgdaPrimitive{𝓤₀}\AgdaSpace{}%
\AgdaPrimitive{𝓤₀}\AgdaSpace{}%
\AgdaBound{𝓤}\AgdaSymbol{)}\AgdaSpace{}%
\AgdaSymbol{→}\AgdaSpace{}%
\AgdaBound{𝓤}\AgdaSpace{}%
\AgdaOperator{\AgdaFunction{̇}}\<%
\\
%
% \\[\AgdaEmptyExtraSkip]%
% %
% \>[2]\AgdaKeyword{module}\AgdaSpace{}%
% \AgdaModule{\AgdaUnderscore{}}\AgdaSpace{}%
% \AgdaSymbol{\{}\AgdaBound{𝓤}\AgdaSymbol{\}}\AgdaSpace{}%
% \AgdaSymbol{\{}\AgdaBound{T}\AgdaSpace{}%
% \AgdaSymbol{:}\AgdaSpace{}%
% \AgdaRecord{AlgTheory}\AgdaSpace{}%
% \AgdaPrimitive{𝓤₀}\AgdaSpace{}%
% \AgdaPrimitive{𝓤₀}\AgdaSpace{}%
% \AgdaBound{𝓤}\AgdaSymbol{\}}\AgdaSpace{}%
% \AgdaKeyword{where}\<%
% \\
% \>[2][@{}l@{\AgdaIndent{0}}]%
% \>[4]\AgdaKeyword{open}\AgdaSpace{}%
% \AgdaModule{AlgTheory}\AgdaSpace{}%
% \AgdaBound{T}\AgdaSpace{}%
% \AgdaKeyword{renaming}\AgdaSpace{}%
% \AgdaSymbol{(}sig \AgdaSymbol{to} Σ\AgdaSymbol{)}\<%
% \\
% %
% \\[\AgdaEmptyExtraSkip]%
% %
% \>[4]\AgdaKeyword{postulate}\AgdaSpace{}%
% \AgdaComment{-- HIT 0-constructor}\<%
% \\
% \>[4][@{}l@{\AgdaIndent{0}}]%
\>[6]\AgdaPostulate{node}%
\>[13]\AgdaSymbol{:}\AgdaSpace{}%
\AgdaSymbol{(}\AgdaBound{s}\AgdaSpace{}%
\AgdaSymbol{:}\AgdaSpace{}%
\AgdaField{sym}\AgdaSpace{}%
\sig\AgdaSymbol{)}\AgdaSpace{}%
\AgdaSymbol{(}\AgdaBound{α}\AgdaSpace{}%
\AgdaSymbol{:}\AgdaSpace{}%
\AgdaField{ar}\AgdaSpace{}%
\sig\AgdaSpace{}%
\AgdaBound{s}\AgdaSpace{}%
\AgdaSymbol{→}\AgdaSpace{}%
\AgdaPostulate{FreeAlgebra}\AgdaSpace{}%
\AgdaBound{T}\AgdaSymbol{)}\AgdaSpace{}%
\AgdaSymbol{→}\AgdaSpace{}%
\AgdaPostulate{FreeAlgebra}\AgdaSpace{}%
\AgdaBound{T}\<%
\\
\>[6]\AgdaPostulate{eq}%
\>[13]\AgdaSymbol{:}\AgdaSpace{}%
\AgdaSymbol{∀}\AgdaSpace{}%
\AgdaSymbol{\{}\AgdaBound{t}\AgdaSpace{}%
\AgdaBound{u}\AgdaSymbol{\}}\AgdaSpace{}%
\AgdaSymbol{→}\AgdaSpace{}%
\AgdaField{eqs}\AgdaSpace{}%
\AgdaBound{t}\AgdaSpace{}%
\AgdaBound{u}\AgdaSpace{}%
\AgdaSymbol{→}\AgdaSpace{}%
\AgdaFunction{node*}\AgdaSpace{}%
\AgdaBound{t}\AgdaSpace{}%
\AgdaOperator{\AgdaDatatype{==}}\AgdaSpace{}%
\AgdaFunction{node*}\AgdaSpace{}%
\AgdaBound{u}\<%
\\
%
\>[6]\AgdaPostulate{quot}%
\>[13]\AgdaSymbol{:}\AgdaSpace{}%
\AgdaFunction{is-set}\AgdaSpace{}%
\AgdaSymbol{(}\AgdaPostulate{FreeAlgebra}\AgdaSpace{}%
\AgdaBound{T}\AgdaSymbol{)}\<%
\end{code}

This HIT comes with an iteration principle for \AgdaBound{T}-algebras and an
induction principle.
These can be used to show that \FreeAlg{\AgdaBound{T}} is indeed the initial
\AgdaBound{T}-algebra.
Details and examples can be found here:
\url{https://perso.ens-lyon.fr/henning.basold/code/AlgTheoryHIT}.

Why would we be interested in such a construction of free algebras?
First of all, we wish to formalise parts of universal algebras, like free and
quotient algebras, and higher inductive types seem the way to go.
Moreover, once we can construct free algebras, we can also consider inductive
and coinductive types over those free algebras as extension of strictly
positive types.
In particular, semantics of finitely branching transition systems could be
obtained in the final coalgebra for the finite powerset functor, which is
the free join-semilattice.
Finally, we aim to extend the construction to HITs that are not just sets, but
groupoids.
This would enable us to, for example, construct free symmetric monoidal
categories.

In the talk, we will present the algebras and the induction scheme in more
detail, show applications of the above construction, and discuss future
directions.

% Future work:
% \begin{itemize}
% \item Adjunction between sets and algebras that factors through quotient for
%   finitary signatures (lifting quotient to algebras?)
% \item Wider class of inductive/coinductive types; allow free algebra is
%   input/output type
% \item Move from sets to groupoids (free symmetric monoidal category example)
% \end{itemize}

%\bibliographystyle{plain}
%\bibliographystyle{alpha}
%\bibliographystyle{unsrt}
\bibliographystyle{abbrv}
\bibliography{AlgTheoriesHIT}

\end{document}
